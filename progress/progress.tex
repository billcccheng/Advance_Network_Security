\documentclass[12pt, a4paper]{article}

\usepackage{syntonly}
%\syntaxonly
\usepackage{amsmath}
\usepackage{url}
%------------------------------------------------------------------------------
%   page
%------------------------------------------------------------------------------
\usepackage{geometry}
\geometry{margin=1in}

%\pagestyle{empty}

%------------------------------------------------------------------------------
%   paragraph
%------------------------------------------------------------------------------
\usepackage{titlesec}
\setlength{\parindent}{0pt}
%\setlength{\parskip}{5ex plus 0.5ex minus 0.2ex}

%\titleformat{\paragraph}
%    {\normalfont\normalsize\bfseries}{\thesubparagraph}{1em}{}
%\titlespacing*{\paragraph}{\parindent}{3.25ex plus 1ex minus .2ex}{.75ex plus .1ex}
%\titleformat{\paragraph}
%    {\normalfont\normalsize\bfseries}{\theparagraph}{1em}{}
%\titleformat{\subparagraph}
%    {\normalfont\normalsize\bfseries}{\thesubparagraph}{1em}{}
%\titlespacing*{\subparagraph}{\parindent}{3.25ex plus 1ex minus .2ex}{.75ex plus .1ex}

%------------------------------------------------------------------------------
%   algorithm
%------------------------------------------------------------------------------
\usepackage{algorithm, algorithmicx}
\usepackage[noend]{algpseudocode}

%\newcommand*\Let[2]{\State #1 $\gets$ #2}
%\algnewcommand\Return[1]{\State #1}
%\algrenewcommand\algorithmicrequire{\textbf{Precondition:}}
%\algrenewcommand\algorithmicensure{\textbf{Postcondition:}}
%\algrenewtext{EndFor}{}

%------------------------------------------------------------------------------
%   theorem
%------------------------------------------------------------------------------
\usepackage{amsthm}
\newtheorem{theorem}{Theorem}

%------------------------------------------------------------------------------
%   graph
%------------------------------------------------------------------------------
\usepackage{tikz}


%------------------------------------------------------------------------------
%   list
%------------------------------------------------------------------------------
\usepackage{enumitem}


%------------------------------------------------------------------------------
%   paste code
%------------------------------------------------------------------------------
\usepackage{listings}
%\lstset{showspaces=false}
%\lstdefinestyle{showspaces=false}
\usepackage{courier}
%\lstset{basicstyle=\footnotesize\ttfamily,breaklines=true}
\lstset{basicstyle=\normalfont\ttfamily,breaklines=true,showspaces=false}
%\lstset{framextopmargin=50pt,frame=bottomline}

%------------------------------------------------------------------------------
%   url
%------------------------------------------------------------------------------
\usepackage{hyperref}

%------------------------------------------------------------------------------
%   pic
%------------------------------------------------------------------------------
\usepackage{graphicx}
\DeclareGraphicsExtensions{.pdf,.png,.jpg}
%\graphicspath{ {./image} }
%\graphicspath{ {/Users/jjjj222/Documents/Dropbox/2015\_fall/database/project\_1/image} }
%------------------------------------------------------------------------------
%   title
%------------------------------------------------------------------------------
\usepackage{titling}
\setlength{\droptitle}{-5em}
\title{
    Advanced Networking and Security\\
    - Project Progress Report 
    \vspace{-2ex}
}
\author{
    \normalfont \normalsize 
    Chia-Cheng (Jeremy) Tso, 
    424008965\\
    %\vspace{-5ex}
    \normalfont \normalsize 
    Chun-Chan (Bill) Cheng, 
    924000036
    %\vspace{-5ex}
}
\date{
    \normalfont \normalsize 
    %\vspace{-9ex}
    \vspace{-5ex}
}

\begin{document}
%------------------------------------------------------------------------------
%   begin
%------------------------------------------------------------------------------
\maketitle
\section{Introduction}
%After suvery and discussion,
Web application is powerful, but it also requires
extra efforts to get it right.% in terms of security.
To reduce the burden of developors,
we proposed the following basic yet important
steps which would thwart most of low-cost attacks efficiently:
%Vcorrect configuration of system and server,

%------------------------------------------------------------------------------
%   
%------------------------------------------------------------------------------
\section{Configuration}
it's tedious, but it's crucial.
3 parts: system (linux), web server (apache), language
because it's the most common setup.
%Why? it's the basic
%First and formost, the most important one is the setup of environment.

\subsection{general} % may remove?
1. use good account name and password
for both server and application admin

\subsection{system setup} % TODO: better term?
2. never chmod 777
avoid information leak

3. close unnecessary ports/service
check command: netstat, nmap IP
close command: TODO

4. setup ip whitelist
force to connect by vpn
file: /etc/hosts.allow
content: in.telnetd: 192.168.1.0/255.255.255.0, : Allow

5. setup log and backup

%\subsection{admin setup}
%DO's and DON'Ts
%\begin{enumerate}[label=\textbf{\arabic*.}]
%    \item
%\end{enumerate}
%
%\subsection{chmod}

\subsection{web server setup}
TODO: location of setup file

apache:
1. Protecting System Settings
2. Protect Server Files by Default
3. setup log
%http://www.petefreitag.com/item/505.cfm
Hide the Apache Version number, and other sensitive information. %TODO: paraphrase
Make sure apache is running under its own user account and group %TODO: paraphrase
Ensure that files outside the web root are not served
Turn off CGI execution
Make sure only root has read access to apache's config and binaries
%http://www.tecmint.com/apache-security-tips/
Disable Directory Listing
Disable Apache’s following of Symbolic Links

\subsection{language setup}
php.ini?
django?

%------------------------------------------------------------------------------
%   
%------------------------------------------------------------------------------
\section{String I/O}
%most problems are here
TODO: cite
37\% Cross-site scripting
16\% SQL injection
it's error-prone
blah blah

\subsection{Input}
never trust user

\subsubsection{White List}
good for account password
con: limited
1. manual
2. tool

\subsubsection{Escape}
pro: more general,
con: hard to do it right
escape function:
1. php
2. python
%    \item python escape function
%\begin{enumerate}[label=\textbf{\arabic*.}]
%    \item php escape function
%    \item python escape function
%\end{enumerate}

\subsection{Output}
reason: avoid XSS
method: escape?
how to escape html?


%------------------------------------------------------------------------------
%   
%------------------------------------------------------------------------------
\section{Database Access}
Almost all the information are stored in database today,
including users' account/password, their data, and their
personally identifiable information (PII).
Therefore,
datacases are often the main target of attackers.
Among all the atacks, SQL injection is the most common one.
We focus on MySQL here because it's the LAMP \cite{LAMP} boudle is widely used.
\\\\
%\subsection{setup}
%account, password, access
%information.schema?
%access permission?
Setting up database correctly, as other part of the system, is a necessity.
We will discuss that later.
Another key point is to use the specialized functions provided by databases
instead of general ones. For example, using proposed function can effectively
avoid some SQL injection. Below is an exmple in Python:
\\\\
%\subsection{R/W}
%use proposed: avoid sql injection
%python example:
%cite: https://www.python.org/dev/peps/pep-0249/ Python Database API
%.execute ( operation [, parameters ])
%Prepare and execute a database operation (query or command).
%cur example: http://initd.org/psycopg/docs/usage.html#query-parameters
Don't:
\begin{lstlisting}[language=python]
cursor.execute("SELECT * FROM table WHERE id=%s" % (id));
\end{lstlisting}
Do:
\begin{lstlisting}[language=python]
cursor.execute("SELECT * FROM table WHERE id=%s" , id);
\end{lstlisting}
The first one is normal format string construction in python.
placeholder \%s will be replaced by id,
and the whole query will be executed.
Apparently, there is an injection problem.
%injection problem
On the other hand,
the second one use proposed function with paramters.
It pre-computes the query and apply the whole \%s directly for comparison.
%pre-compute
If somebody tries to perform an SQL injection, the system will block it automatically.
\begin{lstlisting}[language=python]
Warning: Truncated incorrect DOUBLE value: '1 OR 1=1'
\end{lstlisting}


%\subsection{data}
%django
%https://docs.djangoproject.com/en/1.8/topics/auth/passwords/
%http://stackoverflow.com/questions/749682/django-passwords Django Code

%https://en.wikipedia.org/wiki/Rainbow_table Rainbow table
%1. hash passward
%2. add salt while hashing, rainbow table


%------------------------------------------------------------------------------
%   crypto
%------------------------------------------------------------------------------
\section{Cryptographic Utilities}
%Needless to say, the
%network
Everybody knows that it's important to encrypt data while sending them over the internet,
yet not everybody does it right.
As the technologies of cryptography evolving,
they become more and more complicated.
%It's not the simple one-key-one-algorithm anymore.
Normally, in an encrpyted communication, there are more the 1 keys involved,
along with parameters, nonces, and different kinds of operation modes;
some of them should be kept secret, while others are not;
some of them can be reuse, while others must change every time.
%some are public, some should be kept secret,
%set up, %nonce, %some
It's overwhelming for most of developors, and even an expertise might sometimes get it wrong.
\\\\
%key, function, implementatino, nonce.
One of the exmples is the famous Sony PlayStation 3 issue \cite{PS3_Jailbreak}:
Sony used ECDSA \cite{ECDSA} but didn't change the ephemeral key, so the
private was calculated easily.
%cite https://en.wikipedia.org/wiki/PlayStation_Jailbreak
%cite https://en.wikipedia.org/wiki/Elliptic_Curve_Digital_Signature_Algorithm
Other well-known example is the
side channel attack such as differential power analysis \cite{differential}:
%cite https://en.wikipedia.org/wiki/Side-channel_attack
The implementation is correctly, but functional correctness is not sufficient to
be crpytographically secure.
%For the
%lack of knowledge
%looks fine but indeed not
\\\\
%cite http://www.nist.gov/ NIST
%cite https://www.owasp.org/index.php/Guide_to_Cryptography OWASP guide
%cite http://csrc.nist.gov/groups/ST/toolkit/documents/dss/NISTReCur.pdf NIST ecc recommendation
To support developors, we concluded the following guildlines:
%Three most important concepts
\begin{enumerate}[label=\textbf{\arabic*.}]
    \item
Always encrypt the communication. You never know what information would help hackers
break into your syatem, even if they appear to be harmless.
    \item
Use well-known standard, don't invent the encrpytion algorithm.
%TODO: NIST \cite{NIST}
%TODO: NXP mifare \cite{mifare_break}
    \item
%use the good one
Use the up-to-date standard.
Some algorithms such as md5 have been proved to be unsafe \cite{md5_break}.
    \item
Use libraries, don't implement algorithm yourself.
(even  RNG)
    \item
Use it right,
follow guides carefully for setup (such as those published by NIST \cite{NIST}).
\end{enumerate}

We will provide a more completet list of recommended functions,
for hash, symmetric and assymetric encropyion,
in many common web-debelop languages
in the final report.

%\subsection{SHA, AES, RSA, ECC?}
%provide some python function
%%cite: https://docs.python.org/2/library/hashlib.html#module-hashlib Python hashlib
%
%%cite: https://en.wikipedia.org/wiki/Cryptographic_hash_function#Cryptographic_hash_algorithms
%Don't use MD5, SHA-1
%Use SHA-2
%%cite: https://en.wikipedia.org/wiki/SHA-2
%
%good function for
%1. hash
%2. encrpytion
%3. randeon number geneerator
%%https://pypi.python.org/pypi/pycrypto
%python crypto function
%
%key strength match
%because the weakest part
%
%\subsection{SSL}
%library and their setup?


%------------------------------------------------------------------------------
%   
%------------------------------------------------------------------------------
\section{Useful Tools}
Many tools such as packer or fuzzer can help developers
further improve the quality of web application.
We will introduce some tools after doing some experiments on them.
%use tool to help us

%\subsection{packer}
%1. harden code
%2. enhance runtime
%\subsection{fuzzing/penetration test?}
%1. test

%\section{TDOO}
%1. build case to test
%2. detail usage
%3. test whether the guildlines are truly useful

\bibliographystyle{unsrt}
\bibliography{references}
%------------------------------------------------------------------------------
%   end
%------------------------------------------------------------------------------
\end{document}
