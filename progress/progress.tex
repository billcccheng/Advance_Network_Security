\documentclass[12pt, a4paper]{article}

\usepackage{syntonly}
%\syntaxonly
\usepackage{amsmath}
\usepackage{url}
%------------------------------------------------------------------------------
%   page
%------------------------------------------------------------------------------
\usepackage{geometry}
\geometry{margin=1in}

%\pagestyle{empty}

%------------------------------------------------------------------------------
%   paragraph
%------------------------------------------------------------------------------
\usepackage{titlesec}
\setlength{\parindent}{0pt}
%\setlength{\parskip}{5ex plus 0.5ex minus 0.2ex}

%\titleformat{\paragraph}
%    {\normalfont\normalsize\bfseries}{\thesubparagraph}{1em}{}
%\titlespacing*{\paragraph}{\parindent}{3.25ex plus 1ex minus .2ex}{.75ex plus .1ex}
%\titleformat{\paragraph}
%    {\normalfont\normalsize\bfseries}{\theparagraph}{1em}{}
%\titleformat{\subparagraph}
%    {\normalfont\normalsize\bfseries}{\thesubparagraph}{1em}{}
%\titlespacing*{\subparagraph}{\parindent}{3.25ex plus 1ex minus .2ex}{.75ex plus .1ex}

%------------------------------------------------------------------------------
%   algorithm
%------------------------------------------------------------------------------
\usepackage{algorithm, algorithmicx}
\usepackage[noend]{algpseudocode}

%\newcommand*\Let[2]{\State #1 $\gets$ #2}
%\algnewcommand\Return[1]{\State #1}
%\algrenewcommand\algorithmicrequire{\textbf{Precondition:}}
%\algrenewcommand\algorithmicensure{\textbf{Postcondition:}}
%\algrenewtext{EndFor}{}

%------------------------------------------------------------------------------
%   theorem
%------------------------------------------------------------------------------
\usepackage{amsthm}
\newtheorem{theorem}{Theorem}

%------------------------------------------------------------------------------
%   graph
%------------------------------------------------------------------------------
\usepackage{tikz}


%------------------------------------------------------------------------------
%   list
%------------------------------------------------------------------------------
\usepackage{enumitem}


%------------------------------------------------------------------------------
%   paste code
%------------------------------------------------------------------------------
\usepackage{listings}
%\lstset{showspaces=false}
%\lstdefinestyle{showspaces=false}
\usepackage{courier}
%\lstset{basicstyle=\footnotesize\ttfamily,breaklines=true}
\lstset{basicstyle=\normalfont\ttfamily,breaklines=true,showspaces=false}
%\lstset{framextopmargin=50pt,frame=bottomline}

%------------------------------------------------------------------------------
%   url
%------------------------------------------------------------------------------
\usepackage{hyperref}

%------------------------------------------------------------------------------
%   pic
%------------------------------------------------------------------------------
\usepackage{graphicx}
\DeclareGraphicsExtensions{.pdf,.png,.jpg}
%\graphicspath{ {./image} }
%\graphicspath{ {/Users/jjjj222/Documents/Dropbox/2015\_fall/database/project\_1/image} }
%------------------------------------------------------------------------------
%   title
%------------------------------------------------------------------------------
\usepackage{titling}
\setlength{\droptitle}{-5em}
\title{
    Advanced Networking and Security\\
    - Project Progress Report 
    \vspace{-2ex}
}
\author{
    \normalfont \normalsize 
    Chia-Cheng (Jeremy) Tso, 
    424008965\\
    %\vspace{-5ex}
    \normalfont \normalsize 
    Chun-Chan (Bill) Cheng, 
    924000036
    %\vspace{-5ex}
}
\date{
    \normalfont \normalsize 
    %\vspace{-9ex}
    \vspace{-5ex}
}

\begin{document}
%------------------------------------------------------------------------------
%   begin
%------------------------------------------------------------------------------
\maketitle
\section{Introduction}
After suvery and discussion, we concluded the following simple yet important
steps which can thwart most of the basic attacks.

\section{Configuration}
Why? it's the basic
First and formost, the most important one is the setup of environment.
\subsection{system setup}
1. don't use easy account name and password
2. never chmod 777
3. close unnecessary ports/service
command: netstat, nmap IP
4. setup whitelist
file: /etc/hosts.allow
content: in.telnetd: 192.168.1.0/255.255.255.0, : Allow
5. setup log and backup

%\subsection{admin setup}
%DO's and DON'Ts
%\begin{enumerate}[label=\textbf{\arabic*.}]
%    \item
%\end{enumerate}
%
%\subsection{chmod}

\subsection{web server setup}
apache:
1. Protecting System Settings
2. Protect Server Files by Default
3. setup log
%http://www.petefreitag.com/item/505.cfm
%http://www.tecmint.com/apache-security-tips/
\subsection{language setup}
php?
django?
rail?

\section{String IO}
%most problems are here
TODO: cite
37\% Cross-site scripting
16\% SQL injection
it's crucial
blah blah

\subsection{Input}
never trust user

\subsubsection{White List}
good for account password
con: limited
1. manual
2. tool

\subsubsection{Escape}
pro: more general,
con: hard to do it right
escape function:
1. php
2. python
%    \item python escape function
%\begin{enumerate}[label=\textbf{\arabic*.}]
%    \item php escape function
%    \item python escape function
%\end{enumerate}

\subsection{Output}
reason: avoid XSS
method: escape?

\section{Database Access}
it's the target of attacker
\subsection{setup}
account, password, access

\subsection{R/W}
use proposed: avoid sql injection

\subsection{data}
1. hash passward
2. add salt while hashing, rainbow table

\section{Crypto Function}
\subsection{SSL}
library and their setup?

\subsection{SHA, AES, RSA, ECC?}
good function for
1. hash
2. encrpytion
3. randeon number geneerator

\section{Tools}
use tool to help us
\subsection{packer}
1. harden code
2. enhance runtime
\subsection{fuzzing/penetration test?}
1. test

\section{TDOO}
1. build case to test
2. detail usage
3. test whether the guildlines are truly useful

\bibliographystyle{unsrt}
\bibliography{references}
%------------------------------------------------------------------------------
%   end
%------------------------------------------------------------------------------
\end{document}
